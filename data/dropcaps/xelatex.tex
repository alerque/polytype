\documentclass{article}
\usepackage[paperheight=148.5mm,paperwidth=105mm]{geometry}
\pagenumbering{gobble}
\usepackage{type1cm}
\usepackage{lettrine}
\begin{document}
\lettrine{T}his paragraph has a pretty plain initial or drop cap.
It uses the default document font.
You didn’t really expect more detail with such a generic font choice, right?
This may be exactly what you want, especially with modern typesetting styles which tend towards the minimalist.

“Never say never,” the saying goes.
But someday your dropcap may include leading punctuation {and} a hanging indent.
No worries.
Insert a negative width kern as part of the initial content.
This does double duty: the measured width of the initial will be smaller and we'll back up by that value at the outset.
\end{document}
